\documentclass[12pt]{article}

\usepackage{amssymb,amsmath,amsfonts,eurosym,geometry,ulem,graphicx,caption,color,setspace,sectsty,comment,footmisc,caption,natbib,pdflscape,subfigure,array,hyperref}

\normalem

\onehalfspacing
\newtheorem{theorem}{Theorem}
\newtheorem{corollary}[theorem]{Corollary}
\newtheorem{proposition}{Proposition}
\newenvironment{proof}[1][Proof]{\noindent\textbf{#1.} }{\ \rule{0.5em}{0.5em}}

\newtheorem{hyp}{Hypothesis}
\newtheorem{subhyp}{Hypothesis}[hyp]
\renewcommand{\thesubhyp}{\thehyp\alph{subhyp}}

\newcommand{\red}[1]{{\color{red} #1}}
\newcommand{\blue}[1]{{\color{blue} #1}}

\newcolumntype{L}[1]{>{\raggedright\let\newline\\arraybackslash\hspace{0pt}}m{#1}}
\newcolumntype{C}[1]{>{\centering\let\newline\\arraybackslash\hspace{0pt}}m{#1}}
\newcolumntype{R}[1]{>{\raggedleft\let\newline\\arraybackslash\hspace{0pt}}m{#1}}


\geometry{left=1.0in,right=1.0in,top=1.0in,bottom=1.0in}

\begin{document}

\begin{titlepage}
\title{A review of High Dimensional Nonlinear Data Dimension Reduction Methods
\thanks{Electronic address: \texttt{jaguirre31@gatech.edu}; Corresponding author}}
\author{Jacob Aguirre, Ruirui Ma, Shrey Patel}
\date{\today}
\maketitle
\begin{abstract}
\noindent Massive, high-dimensional data sets are already appearing in a variety of sectors of modern research, posing new obstacles. These high-dimensional data sets can typically be modeled as point clouds in a $D-$dimensional space, such that it's concentrated near a $d-$dimensional manifold. Thus, we are able to visualize this data as a low-dimensional representation. The well-known curse of dimensionality in machine learning suggests that a huge amount of training data is necessary in order to obtain a given prediction accuracy. Unless additional assumptions are established, picture and signal recovery requires a significant number of observations to recover a high-dimensional vector. Owing to the fact there exist rich localized constants, global symmetry, repeated patterns, or redundant sampling, many real-world data sets show low-dimensional geometric structures. This study seeks to investigate these local structures on a variety of simulated and real data-sets, and discover which properties contribute to certain models performing well or under performing. As a result, we're attempting to investigate low-dimensional geometric patterns in data sets in order to extract features, forecast data, and recover signals.  \\
\vspace{0in}\\
\noindent\textbf{Keywords:} High-Dimensional Data, Machine Learning, Statistical Inference\\
\vspace{0in}\\
\noindent\textbf{JEL Codes:} C15, C44, C45\\

\bigskip
\end{abstract}
\setcounter{page}{0}
\thispagestyle{empty}
\end{titlepage}
\pagebreak \newpage




\doublespacing


\section{Background} \label{sec:introduction}

\hspace{5mm}The $21^{st}$ century presents new challenges in the realm of statistical inference and data mining. High dimensional data sets are arising more often, which contributes to challenges in attempting to understand and draw conclusions from such data. For the purposes of image and signal recovery, a large amount of technical expertise is needed in order to recover a high-dimensional vector we're interested in [1].

In addition, number of high school students who used e-cigarette increased by 78\% resulting in 3.05 million and the number of middle school students who use e-cigarettes increased by 48\% bringing the total up to 570,000 between 2017-2018 [3]. Furthermore, as many as one in five U.S. adult cigarette smokers have tried e-cigarettes [4,5]. Current cigarette users could decide to continue smoking cigarettes, or adjust their patterns and begin using both cigarettes \& e-cigarettes, or neither.

Understanding how risk perceptions are formed and how they are influenced by extrinsic health and information shocks is therefore critical to comprehending various usage patterns and developing strategies to correct misperceptions and reduce harmful public health effects. Our research adds to the current limited literature on the subject by providing some of the first assessments of how risk perceptions change in response to a negative information shock and during a phase of extensive government and media messaging about the dangers of vaping.

In this study, I seek to use 2018, 2019, and 2020 National Youth Tobacco Survey data to build different prediction models that can predict whether or not a person would consume e-cigarettes. I divide the data into two groups: never-smokers and cigarette smokers. The never-smokers' data was evaluated to establish the best model for predicting the intention to smoke cigarettes in both e-cigarette and non-e-cigarette smokers.

The rest of this paper is as follows: Section 2 introduces background of existing literature of risk perceptions of e-cigarettes and the effects it has on future usage; Section 3 provides
a theoretical model of the problem and extends an empirical approach to measure the trend in risk perceptions and future usage; Section 4 provides a commentary of the results, as well as supplying an analytic approach to decide whether our hypothesis was correct; Section 5 presents concluding remarks and avenues of future research.


\section{Methodology} \label{sec:methodology}

\hspace{5mm}We sought to implement four different, unique nonlinear dimension reduction techniques on seperate, real-world data sets. The group decided on the following arrangement: Jacob would work on Isomap and Local Linear Embedding, Ruirui implementing the Diffusion map algorithm, and Shrey investigating the properties of the T-SNE method. Each of these methods presents their own set of pro's and con's, which we had to navigate around. Thus, this was not a simple cut-and-paste approach to this paper.  



\section{Local Linear Embedding} \label{sec:Local Linear Embedding}


\subsection{Approach}
\hspace{5mm}We seek to extend a Bayesian learning context for the usage and development of a theoretical model, from which we can test our empirical model on. Consider the set $N$ to be the set of agents in the game. Let $\Omega$ denote the state space of a system, where $\sum\limits_{i=1}^{n}a_i=1$ denotes that the probability of all actions in the state space must sum to one, and we'll assume $\Omega$ is finite. For each player $i\in N$, there exists the set $A_i$ (the set of actions available to player $i$). Each player $i$ holds a prior belief of the state of nature given by a probability measure $p_i$ on $\Omega$. In our case, an agent holds a prior perception of e-cigarettes, which can be either positive, neutral, or negative. This degree will vary and can take on any value within $[-1,1]$. 

At any given point of play during the game, some state $\omega\in\Omega$ is realized, such that a user has gained new information of the EVALI crisis, and has updated their prior beliefs. Thus, we seek to model a players decision making process under uncertainty. Define a profile called $\gamma_i$ of a standard signal function, and denote $\gamma_i(\omega)$ to be the agent observing a signal of a state of nature. Assume the agent has a $k$x$1$ vector denoting the personal characteristics for each agent, which are likely to affect their belief states. Define $T_i$ to be the different types of players. I define $p_i(\gamma_i^{-1}(t_i))>0$ $\forall$ $t_i\in T_i$, where a player $i$ assigns positive prior probability to every member of $T_i$.


\subsection{Results}

\section{Diffusion Map} \label{sec: Diffusion Map}

\subsection{Approach}

\subsection{Results}

\section{T-SNE} \label{sec: T-SNE}
\subsection{Approach}

\subsection{Results}

\section{Isomap} \label{sec: Isomap}
\subsection{Approach}

\subsection{Results}


\section{Discussion} \label{sec:discussion}

\hspace{5mm}Many teens in the United States report being exposed to e-cigarette marketing messages via a number of outlets. Ads for e-cigarettes may impact teenage beginning of e-cigarette usage. According to the findings of this study, exposure to e-cigarette ads can also impact teenagers' risk perception of smoking cigarettes, which is a strong predictor of smoking start (Roditis et al., 2016; Song et al., 2009). Youth who've never-smoked and were exposed to e-cigarette commercials reported considerably decreased perceived hazards of smoking cigarettes. Regulating the platforms and material of e-cigarette marketing should be examined in order to reduce teenage exposure and the potentially negative impacts on nonsmokers. 


\subsection{Limitations}
As a result, shifts in attitudes regarding e-cigarettes do not represent individual shifts in attitudes within the same group of persons. The demographics of both groups, however, were indicative of the general population of the United States. Self-reported data was also employed in this study, which has limitations since social desirability bias may cause individuals to offer an incorrect depiction of their own perceptions [15]. Furthermore, the survey was completely optional and we may have lost responses from both adults and youth who decided taking the survey was not worth effort and time. 

Additionally, the NYTS data contains a relatively restricted set of sociodemographic indicators, as well as little information on socioeconomic or familial history. As these factors are linked to teenage nicotine use, leaving them out might lead to bias [16]. 

\newpage
\begin{thebibliography}{}
\bibitem{First Entry}Fannjiang, Albert, and Wenjing Liao. "Coherence pattern–guided compressive sensing with unresolved grids." SIAM Journal on Imaging Sciences 5.1 (2012): 179-202.

\bibitem{increase news coverage EVALI}Leas, Eric C et al. “Self-reported Cannabidiol (CBD) Use for Conditions With Proven Therapies.” JAMA network open vol. 3,10 e2020977. 1 Oct. 2020, doi:10.1001/jamanetworkopen.2020.20977


\bibitem{Second Entry}Dave, D., Dench, D., Kenkel, D. et al. News that takes your breath away: risk perceptions during an outbreak of vaping-related lung injuries. J Risk Uncertain 60, 281–307 (2020). https://doi.org/10.1007/s11166-020-09329-2

\bibitem{Third Entry}Pearson, Jennifer L., et al. "e-Cigarette awareness, use, and harm perceptions in US adults." American journal of public health 102.9 (2012): 1758-1766.

\bibitem{Fourth Entry}King, Brian A., et al. "Awareness and ever-use of electronic cigarettes among US adults, 2010–2011." Nicotine \& tobacco research 15.9 (2013): 1623-1627.

\bibitem{fifth entry}Krishnasamy, Vikram P et al. “Update: Characteristics of a Nationwide Outbreak of E-cigarette, or Vaping, Product Use-Associated Lung Injury - United States, August 2019-January 2020.” MMWR. Morbidity and mortality weekly report vol. 69,3 90-94. 24 Jan. 2020, doi:10.15585/mmwr.mm6903e2


\bibitem{sixth entry}Brose, Leonie S., et al. "Perceived relative harm of electronic cigarettes over time and impact on subsequent use. A survey with 1-year and 2-year follow-ups." Drug and Alcohol dependence 157 (2015): 106-111.

\bibitem{seventh entry}Amrock, Stephen M., et al. "Perception of e-cigarette harm and its correlation with use among US adolescents." Nicotine \& Tobacco Research 17.3 (2015): 330-336.

\bibitem{eighth entry}Biener, Lois, and J. Lee Hargraves. "A longitudinal study of electronic cigarette use among a population-based sample of adult smokers: association with smoking cessation and motivation to quit." Nicotine \& Tobacco Research 17.2 (2015): 127-133.

\bibitem{ninth entry}McMillen, Robert C., et al. "Trends in electronic cigarette use among US adults: use is increasing in both smokers and nonsmokers." Nicotine \& tobacco research 17.10 (2014): 1195-1202.

\bibitem{tenth entry}Rapp, Joseph L et al. “Changes in E-Cigarette Perceptions Over Time: A National Youth Tobacco Survey Analysis.” American journal of preventive medicine vol. 61,2 (2021): 174-181. doi:10.1016/j.amepre.2021.03.006


\bibitem{eleventh entry}Siahpush, Mohammad et al. “Association of smoking cessation with financial stress and material well-being: results from a prospective study of a population-based national survey.” American journal of public health vol. 97,12 (2007): 2281-7. doi:10.2105/AJPH.2006.103580


\bibitem{sociodemographics e-cigarettes}Simon, Patricia, et al. "Socioeconomic status and adolescent e-cigarette use: The mediating role of e-cigarette advertisement exposure." Preventive medicine 112 (2018): 193-198.

\bibitem{Receptivity to E-cigarette smoking}Pokhrel, Pallav, et al. "Receptivity to e-cigarette marketing, harm perceptions, and e-cigarette use." American journal of health behavior 39.1 (2015): 121-131.

\bibitem{School-level clustering us adolescents}Corsi, Daniel J, and Adam M Lippert. “An examination of the shift in school-level clustering of US adolescent electronic cigarette use and its multilevel correlates, 2011-2013.” Health \& place vol. 38 (2016): 30-8. doi:10.1016/j.healthplace.2015.12.007

\bibitem{Underreporting socioeconomic status}Graham, H., and L. Owen. "Are there socioeconomic differentials in under-reporting of smoking in pregnancy?." Tobacco Control 12.4 (2003): 434-434.

\bibitem{Underreporting smoking status}Vartiainen, E., et al. "Validation of self reported smoking by serum cotinine measurement in a community-based study." Journal of Epidemiology \& Community Health 56.3 (2002): 167-170.

\end{thebibliography}


\section{Funding Statement}
Funding for this study was provided by the Georgia Institute of Technology Presidential Undergraduate Research Award. Georgia Tech held no role in the study design, collection, analysis or interpretation of the data, writing the manuscript, or the decision to submit the paper for publication. No co-author held any accompanying grants for this work. 

\section{Disclaimer}
The findings and conclusions in this report are those of the authors and do not necessarily represent the official position of the Georgia Institute of Technology or any of its affiliated institutions or agencies. This article was prepared while Jacob Aguirre was an undergraduate student at the Georgia Institute of Technology, School of Economics and Mathematics. 

\end{document}